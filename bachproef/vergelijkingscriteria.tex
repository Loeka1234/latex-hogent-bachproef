\chapter{Vergelijkingscriteria}%
\label{ch:vergelijkingscriteria}

\section{Niet-functioneel}

\begin{itemize}
    \item Kostprijs
    \item Performantie
\end{itemize}

Kostprijs en performantie zijn makkelijk meetbaar door de twee pipelines uit te voeren. De data is vergelijkbaar met elkaar dus aan de hand van grafieken kan er gekeken worden welke optie bijvoorbeeld het goedkoopst of snelst is. 

\section{Functioneel}

\begin{itemize}
    \item Mogelijkheid tot debuggen
    \item Source control
    \item Infrastructure as Code (IaC)
\end{itemize}

Bovenstaande functionele vergelijkingscriteria zijn moeilijker te kwantificeren. Toch zijn deze belangrijk om te vergelijken zodat er een keuze gemaakt kan worden tussen Azure Data Factory en Azure Databricks. Er gaat dus gekeken worden welke mogelijkheden beide tools hebben om te debuggen, gebruik te maken van source control en gebruik te maken van Infrastructure as Code (IaC). 
